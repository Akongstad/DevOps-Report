% Blaðsíðustillingar

\usepackage{geometry}

\geometry{
	paper=a4paper, % letterpaper lika til
	top=2.5cm, % Top margin
	bottom=1cm, % Bottom margin
	left=2.5cm, % Left margin
	right=2.4cm, % Right margin
	headheight=0.75cm, % Header height
	footskip=1.5cm, % Space from the bottom margin to the baseline of the footer
	headsep=0.75cm, % Space from the top margin to the baseline of the header
	%showframe, % Uncomment to show how the type block is set on the page
}

%-------------------------------- Íslenska ----------------------------
\usepackage[T1]{fontenc}
\usepackage[utf8]{inputenc}
%\usepackage[icelandic]{babel}

%----------------------------- Stærðfræðipakkar frá AMS ---------------

\usepackage{amsmath, amsfonts, amsthm, amssymb} % Stærðfræðipakkar
\usepackage{braket, nicefrac} % fyrir mengi, brotabrot

% ----------- Fyrir SI Einingar
\usepackage{siunitx}


%------------------------------ Listar/ númeringar -------------------------
\usepackage{enumitem, multicol}

%------------------------------- Fyrir innsetningu mynda --------------
\usepackage{graphicx, float} 
\usepackage{keystroke}
\usepackage{pgfplots}\usepgfplotslibrary{units}
\pgfplotsset{combat=1.18}

% ----------------- Til að teikna/tekka myndir -----------------------------
\usepackage{tikz}
\usepackage{tkz-euclide}
\usetikzlibrary{math}
\usepackage{fourier}
\usetikzlibrary{quotes,angles}
\usepackage{tkz-euclide}
\usetikzlibrary{calc}
%\usetkzobj{all}
\usepackage[siunitx]{circuitikz} %%<---Circuit Diagrams

%%%%%%%%%%%%%%%%%%%%%%%%%%
% Nýtt Matlab viðmót
\usepackage{listings}
\usepackage{fancyvrb}

\def\lstbasicfont{\fontfamily{pcr}\selectfont\normalsize}
\definecolor{mygreen}{RGB}{28,172,0} 
\definecolor{mylilas}{RGB}{170,55,241}
\lstset{language=Matlab,%
	basicstyle={\lstbasicfont},
	breaklines=true,%
	morekeywords={matlab2tikz},
	keywordstyle=\color{blue},%
	morekeywords=[2]{1}, keywordstyle=[2]{\color{black}},
	identifierstyle=\color{black},%
	stringstyle=\color{mylilas},
	commentstyle=\color{mygreen},%
	showstringspaces=false, %without this there will be a symbol in the places where there is a space
	numbers=left,%
	numberstyle={\tiny \color{black}},% size of the numbers
	numbersep=5 pt, % this defines how far the numbers are from the text 
	inputencoding=latin1,
	backgroundcolor = \color{gray!3},
	framexleftmargin= -1 mm,
	frame=none,
	rulesepcolor=\color{blue!30},
	extendedchars=true,
	emph={logical},emphstyle=\color{blue},	
	emph={all,equal, minor, on, off, long, short, bank, rat},emphstyle=\color{mylilas},	
}
\renewcommand\lstlistingname{\textsc{Matlab}}%

\usepackage{tcolorbox}
\tcbuselibrary{skins}
%Hérna vel ég stillingar fyrir ramma sem ég skýri matlabUT
\tcbset{matlabUT/.style={
		enhanced,
		colback=gray!1,
		colframe=gray!30,
		title=Command Window,
		arc=0mm,
		coltitle=black,
		center title, 
		title style={top color=white, bottom color = gray!30},
		grow to left by= -3 mm,
		left= 4 mm,
		grow to right by=0.5mm,
		colupper = gray!70!black
}}

% Les inn textaskra sem inniheldur niðurstöður úr Command Window
\newcommand{\CommandWindow}[1]{\begin{tcolorbox}[matlabUT]
		\VerbatimInput{#1}
\end{tcolorbox}}

%%%%%%%%%%%%%%%%    Matlab endar %%%%%%%%%%%%%%%%%%%%%%%%%%%%%%%%%%%%%%%%

% Formatta kóða

 \DefineVerbatimEnvironment%
      {verbatimprog}%
      {Verbatim}%
      {fontsize=\tiny}%

%%%%%%%%%%%%%%%%%%%%%%%%%% Hyperlink References %%%%%%%%%%%%%%%%%%%%%%%%%%%
\usepackage{hyperref}

%--------------------% Storage Path for images %-----------------%
\graphicspath{{graphics/}{Graphics/}{./}}