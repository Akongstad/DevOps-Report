\section{System's Perspective}
\label{sec:systems_perspective}

\subsection{Overview}
\label{subsec:systems_perspective_overview}
The initial project was a full stack Flask application written in Bash and Python2 with an SQLite database. 
This implementation (henceforth referenced as TBD-MiniTwit) included refactoring the initial project to the following containerized microservices:
\begin{itemize}
    \item C\# backend using the ASP.NET web framework and EF Core ORM
    \item ReactJS SPA frontend
    \item PostgreSQL database
\end{itemize}
upon which a monitoring and (temporarily) a logging stack were added to be served as a containerized application behind a load-balancer on a managed Kubernetes cluster.

\subsection{System Design}
\label{subsec:system_design}
% - Design of your ITU-MiniTwit systems
The heart of TBD-MiniTwit is the C\# backend.
It contains all our business logic and provides two open APIs, an Object-Relational Mapping to our database and exposes application-specific metrics for Prometheus. \\
CLASS DIAGRAMS FOR BACKEND \\
STATE MACHINE DIAGRAM FOR LOGIN \\

\subsection{System Architecture}
\label{subsec:system_architecture}
% - Architecture of your ITU-MiniTwit systems
% - All dependencies of your ITU-MiniTwit systems on all levels of abstraction and development stages.
% - That is, list and briefly describe all technologies and tools you applied and depend on.
This implementation follows the \textbf{Microservice Architecture Pattern} \\
EXPLAIN THE PATTERN \\
CREATE DIAGRAMS \\


\subsection{Subsystem Interactions}
\label{subsec:subsystem_interactions}
% - Important interactions of subsystems
\begin{table}[h!]
    \centering
    \begin{tabular}{|c|} \hline
         Prometheus \\ \hline
         MinitwitAPI \\ \hline
         LoadBalancer \\ \hline
    \end{tabular}
    \caption{Network interfaces}
    \label{tab:my_label}
\end{table}

\subsection{System state}
\label{subsec:system_state}
% - Describe the current state of your systems, for example using results of static analysis and quality assessment systems.


\subsection{License Compatibility}
\label{subsec:license_compatability}
% - Finally, describe briefly, if the license that you have chosen for your project is actually compatible with the licenses of all your direct dependencies.
The project uses MIT license meaning everyone can use and modify.

% A description and illustration of the:
% - Double check that for all the weekly tasks (those listed in the schedule) you include the corresponding information.